\documentclass[a4paper,10pt]{article}
\usepackage[utf8]{inputenc}
\usepackage{hyperref}

%opening
\title{Auslagerung der Ausführung von Methoden der HYPRE Bibliothek in ein Cloudsystem
Recherche und Literaturverzeichnis}
\author{Thomas Rückert}

\begin{document}

\maketitle
\newpage

\begin{abstract}

\end{abstract}

\newpage

\tableofcontents

\newpage

\section{Informationssammlung zu Themen für die Arbeit}
\subsection{Recherche zum Thema Cloud}

\begin{itemize}
 \item was ist eine cloud?
 \item welche arten gibt es
 \item welche gibt es
\end{itemize}

\url{https://aws.amazon.com/types-of-cloud-computing/}

\subsection{HYPRE - Überblick über die Bibliothek}

\begin{itemize}
 \item metainfos
 \item sprache(n)
 \item performance
 \item welche probleme löst die bib
 \item 
\end{itemize}

\newpage

\section{Vergleich verschiedener Technologien und Werkzeuge für den Einsatz bei der Implementierung}

\subsection{'private' vs 'public' Cloud}

\begin{itemize}
 \item gibt es open source cloud systeme? welche?
 \item welche public clouds gibt es (zB aws)
 \item was sind unterschiede (neben dem access, zB performance?)
\end{itemize}

\subsection{Technologie}

\begin{itemize}
 \item welche Werkzeuge für die kommunikation zwischen client und cloud
 \item rpc, socket, service ...
 \item abwägen zwischen performance, aufwand ...
 \item welche framework könnten genutzt werden
\end{itemize}


\subsection{Cloudtyp}

wird evtl schon teilweise in abschnitt 1 (allgemeines zur cloud) abgedeckt

\begin{itemize}
 \item Infrastructure as a Service (IaaS)
 \item Platform as a Service (PaaS)
 \item Software as a Service (SaaS)
 \item hybrid
 \item wie eignen sich diese typen für `unsere` Implementierung
\end{itemize}

\subsection{Sprache}

\begin{itemize}
 \item hardware-nah: C/C++ (bessere performance)
 \item vs netzwerknah: (bessere möglichkeiten die kommunikation zu Implementieren)
 \item kann eine hybridform eingsetzt werden? zB (micro-)service: bib in c kommuniziert mit client-backend in php, dieses führt die calls zum server aus
\end{itemize}

\newpage

\section{Zukünftige, weiterführen Arbeiten}

eventuell werden Teile von hier in die aktuelle Arbeit verschoben

\begin{itemize}
 \item Skalierbarkeit der Cloud (einsetzen)
 \item Vorteile der Auslagerung:
  \begin{itemize}
    \item performance
    \item speicher
    \item lösbarkeit (nur remote überhaupt lösbar)
  \end{itemize}

\end{itemize}


\end{document}
